\documentclass[12pt, a4paper]{article}
\usepackage[left=4cm, right=6cm, top=2cm, bottom=2cm]{geometry}
\usepackage[english]{babel}
\usepackage[affil-it]{authblk}

\providecommand{\keywords}[1]{\let\thefootnote\relax\footnote{Keywords: #1}}

\renewcommand*{\Authands}{, }
\renewcommand*{\Affilfont}{\small\it}

\title{Regular Formal Modules in One-dimensional Local Fields \thanks{The article is written with support of RFBR grant \#14-01-00393.}}
\author[1]{S. M. Vlassiev}
\author[2]{S. V. Vostokov}
\author[1]{A. A. Gorshkov}
\affil[1]{Graduated student at Mathematics and Mechanics faculty, St. Petersburg State University}
\affil[2]{Professor at Mathematics and Mechanics faculty, St. Petersburg State University}
\date{}

\begin{document}
\maketitle
\keywords{local field, totally regular local field, formal module, formal group}

\begin{abstract}
This article studies local fields which doesn't contain non-trivial roots of isogeny of formal group over its ring of integers together with its unramified extensions that also doesn't contain such roots. Originally this problem appeared in 1962 in Z. I. Borevich's article during work on tamely ramified extensions in multiplicative formal groups case.
\end{abstract}
\end{document}